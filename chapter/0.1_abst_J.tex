\begin{center}
\hrulefill\\
\vskip\baselineskip
{\bf\Large KAGRAにおける低温懸架装置の特性評価と制御} \addcontentsline{toc}{chapter}{要旨} \\
\large{                      東京大学理学系研究科物理学専攻 玉木諒秀}\\
\large{                                 \,\,\,指導教員 三代木伸二}\\
\vskip\baselineskip
{\bf \Large 要旨}
\end{center}
{\fontsize{13.5pt}{45pt}\selectfont
\quad 2015年9月, アメリカの重力波望遠鏡LIGOが初めて重力波の直接検出に成功した. それ以降もLIGOおよびイタリアの重力波望遠鏡 Virgo によって共同観測が行われて国際重力波観測ネットワークが構築され, 2023年1月現在, 約90の重力波イベントが検出されている. こうして開幕した重力波天文学は, 国際重力波観測ネットワークがさらに広がることによって, 今後も発展していくと期待されている. また, 日本の重力波望遠鏡KAGRAも国際重力波観測ネットワークへ参入し, 波源の方向決定精度の向上や重力波の偏極の観測に貢献することを目指している.  2020年4月にドイツの重力波望遠鏡GEO 600と初めて重力波の共同観測 (O3GK) をおこなったKAGRAは2023年3月, LIGO・Virgoと共同観測 (O4) を行う予定である.\\
\quad KAGRAは熱雑音低減のためにサファイア鏡を20 Kまで冷却するという, これまでの重力波検出器にはない特徴を持っており, O4では10 Mpcという目標感度達成のために鏡を冷却して観測を行う. しかし, 鏡が常温のまま観測を行ったO3GKと異なり, O4では鏡の冷却によって鏡を吊るす懸架装置の物性が変化し得る. このため, 常温で動作していた制御が低温でも動作するかどうかは自明ではない. また, 干渉計の安定な動作のためには共振周波数での振動を抑えるダンピング制御が必要であるが, O3GKでは$10\sim50$ Hzの低周波帯において, サファイア鏡を吊るす低温懸架装置の制御系由来の雑音が検出器の感度を制限していた. この周波数帯は中性子星連星からの重力波のうち, インスパイラルフェイズの周波数に含まれ, 中性子星の質量を得るための解析に用いられる. こうして得られた中性子星の質量は, 合体フェイズ以降の重力波から得られる中性子星の構造と合わせることで, 中性子星物質の状態方程式に制限をつけることができると考えられている. それが実現すれば天文学や宇宙物理学, さらには原子核物理や素粒子物理など幅広い分野の発展につながる. その実現のためにもO4では低周波における制御雑音の低減が求められる.\\
\quad このような背景を踏まえ, 以下の通り研究を行なった. まず2022年12月現在, 82 Kまで冷却された低温懸架装置の1つに対してその特性(懸架装置の共振周波数やQ値, 制御系のためのフォトセンサの出力や伝達関数)が常温と比べてどのように異なるかを調べた. また, 低温懸架装置に対してKAGRAにおける振動減衰の要求を満たすようなダンピング制御を設計した上で, それが低温でも成り立つかどうか調べた. さらに, 観測段階で用いるための低雑音な制御フィルタ(Observation フィルタ)を新たに設計し, 実装した.\\
\quad その結果, 冷却に伴う低温懸架装置の異常は見受けられなかった. また, 常温において振動減衰の要求を満たすように設計したダンピング制御が, ゲインの変更さえすれば低温でも成り立つことを確認した. さらに, 新たに設計・実装したObservationフィルタを用いることで, $10\sim50$ Hzにおける低温懸架装置の制御雑音を$2\sim3$ 桁低減し,  その上で検出器の安定な動作に成功した.}

\clearpage