修士課程での研究活動および本論文の執筆にあたり, 多くの方々にご支援いただきました. ご協力いただいた全ての皆様に深く感謝申し上げます. \\\\
\quad 指導教官である三代木伸二教授には修士課程中のあらゆる面でお世話になりました. 研究や発表, 論文の執筆などの進捗を気にかけてくださったほか, 助言も数多くいただきました. そればかりでなく, 旅費や送迎をはじめ, 神岡で研究活動に取り組めるよう様々な手配をしていただきました. 何の気兼ねもなく神岡で研究ができたことに感謝しています. 氏の学生であることを誇りに思い, それに恥じない研究生活を送りたいと思います. \\\\
\quad 宇宙線研究所の牛場崇文助教授には2年間を通じて大変お世話になりました. 進学前であったにも関わらず, KAGRAにおける低温技術の説明をしていただき, さらに興味が深まったのを覚えています. また, 神岡では坑内作業の際にご一緒させていただき, 懸架装置の仕組みやソフトウェアの使い方などを1から丁寧に教えてくださいました. さらに学会発表や本論文についても丁寧に見ていただき, その度に的確なコメントを数多くいただきました. 本論文をまとめることができたのは氏のおかげです.\\\\
\quad 宇宙線研究所の内山隆准教授は何度も富山駅から神岡まで送迎していただいたほか, 体調や研究の進捗など, いつも気にかけていただきました. また, 毎週金曜日には研究室ミーティング及びセミナーを執り行っていただいたことに加え, 学会などの発表練習にも付き合っていただき, その度に的確かつ鋭い質問をしていただきました.   \\\\
\quad 国立天文台の都丸隆行教授にはあらゆる面で大変お世話になりました. 坑内作業でご一緒させていただく際は, その手際の良さと実験の深い知識から多くを学びました. また, 低温技術について雑談したり相談させていただく際は, 魅力的なアイデアが次々と出てきて驚かされただけでなく, 研究を行う上での物事の考え方を数多く示していただき非常に参考になりました. ご自身の研究室の学生ではないにも関わらず, KEKの見学やTAに至るまで, あらゆる場面で気にかけていただき, 大変感謝しております. \\\\
\quad 富山大学の山元一広准教授からは主にCRYグループのミーティングにおいて, 研究に対する助言やコメントを数多くいただきました. さらに, 学会発表やポスター発表の資料および本論文に対しても的確なご指摘やアドバイスをくださりました. また, 重力波実験の分野で低温に関する研究をするにあたり, 氏の修士論文は大変参考になりました. \\\\
\quad 宇宙線研究所の大橋正健教授は神岡で研究する際に不自由がないよう, いつも気にかけてくださいました. 氏のおかげで神岡での研究が快適なものになりました.\\\\
\quad 宇宙線研究所の宮川治准教授は学会や修士論文審査会の発表練習に付き合っていただき, スライドの見せ方から話し方に至るまで, 発表をわかりやすくするための方法を一緒になって考えてくださいました.   \\\\
\quad 宇宙線研究所の山本尚弘助教授はコントロールルームにいらっしゃることが多く, デジタルシステムについて分からないことがあるとその都度丁寧かつ分かりやすく教えてくださいました. 毎日遅くまで残って研究活動に取り組む氏の姿勢からは, 学ぶべきところがたくさんありました. \\
\quad 宇宙線研究所の横澤孝章特任助教授は深夜から朝方にかけての作業において測定の手伝いをしていただき, また本論文にもご意見をくださいました. さらに, ご自身の経験をもとに,大学院生として活動する上での助言を数多くしていただきました. \\
\quad 宇宙線研究所の譲原浩貴特任助教授には主に, noise budgetのツール関連でお世話になりました. 夜遅い時間でも質問・提案に応えていただき、大変助かりました. また, いつも明るい氏との会話は研究や修士論文執筆の息抜きでありました.\\
\quad 宇宙線研究所の押野翔一特任助教授にはKAGRAにおけるDACについて教えていただきました. また修士論文の進捗も気にかけてくださいました.\\
\quad 国立天文台の鷲見貴生特任助教授には書類を添削していただいたり, 大学院生および研究者の生活について色々と教えていただきました.\\\\
\quad 宇宙線研究所技術専門員の戸村友宣氏は研究や生活の様子を気にかけてくださり, また富山市での生活を色々と教えてくださいました.\\
\quad 宇宙線研究所学術専門職員のPENA ARELLANO, Fabian氏はいつも陽気に話しかけてくださったほか, DACノイズの計算で大変お世話になりました. (PENA ARELLANO, Fabian, an academic specialist at the ICRR, was always cheerfully, and he helped me a lot in the DAC noise calculations.)\\
\quad 宇宙線研究所学術専門職員の安居宏実氏, 国立天文台主任技術員の平田直篤氏, 国立天文台特任専門員の池田覚氏は坑内作業でお世話になったほか, 気さくに話しかけてくださいました.\\\\
\quad KEKの鈴木敏一氏, 山田智宏氏, Rishabh Bajpai氏, 宇宙線研究所学術専門職員の野手綾子氏にはCRYグループのミーティングでお世話になったほか, KEKを訪ねた際はの低温施設を案内してくださったりご自身の研究について教えてくださいました.\\\\
\quad 宇宙線研究所事務室の舟田真也氏, 加藤大地氏, 沖中美保子氏, 坂本絵里氏, 野尻みどり氏には事務手続きで大変お世話になった他, 神岡での生活面も気にかけてくださいました. 皆様のおかげで何の不自由もなく神岡に出張できました. \\
\quad また, 菊池理恵氏, 工藤直美氏には主に出張の手続きでお世話になりました. 予約の不備や出張延長・短縮申請などでご迷惑をおかけすることが多々ありましたが, 親切に対応してくださりました. \\\\
\quad 香港中文大学のTSANG, Terrence Tak Lun氏には制御理論について様々なことを教わりました. 正直, 最初の頃は氏の言うことが半分も分かりませんでしたが, 私が理解するまで付き合ってくださりました. 中国に戻ってからもポスター発表の資料にメールでアドバイスをくださり, 大変助かりました. また, 氏の博士論文は非常に興味深く, かつ参考になりました. (TSANG, Terrence Tak Lun from the Chinese University of Hong Kong taught me many things about control theory. To be honest, I couldn't understand half of what he said in the beginning, but he patiently explained to me until I understood. After he returned to China, he helped me by giving advice on my poster presentation by email. Also, his doctoral thesis was very interesting and informative.)\\
\quad 梶田研究室の田中健太氏には干渉計関係の事柄を数多く教えていただきました. また, 遅くまで解析棟に残って研究以外の話題で盛り上がることも多々ありました. 博士論文を抱えているにも関わらず, 常に落ち着いた様子で後輩のことも気にかけてくださる氏は非常に頼もしい先輩です. また, 新潟大学の廣瀬千晶氏は研究活動から趣味に至るまで様々な話をしていただきました. 田中氏, 廣瀬氏との会話は修士論文を書く上で息抜きでもあり, 刺激でもあって大変楽しかったです. \\
\quad 内山研究室の千葉廉一氏とは顔を合わせる機会こそ少なかったですが, 神岡で会うたびに色々な話を聞かせてくれました. 研究活動だけでなく, それ以外にも自分のやりたいことを見つけて両方に取り組む氏からは刺激を受けました. また, 都丸研究室の西野耀平氏は研究科は違えど, 氏が神岡に来た際などに会話をし, その都度知識の深さと理解度の高さに驚かされました. 自らの研究に高いレベル, モチベーションで取り組むのは勿論, 様々なことを意欲的に勉強する氏の姿からは多くを学びました. \\
\quad 三代木研究室の藤井慎吾氏, 内山研究室の宮本慎也氏, 山村隼聖氏は対面授業の関係で顔を合わせる機会はそれほどありませんでしたが, セミナーにおける氏らの素晴らしい発表のおかげで, 自分もまだまだ勉強しなければならないと再認識できました. \\\\
\quad ここには書ききれませんが, 上に挙げた方以外にも本当に多くの方々にご支援をいただきました。深く感謝いたします. \\\\
\quad 本論文の結びとして, これまでの生活を支えてくれた両親をはじめとする家族に感謝します. 大学院まで進学させてくれただけでなく, 人生のあらゆる面でサポートしていただきました. 本当にありがとうございました. 