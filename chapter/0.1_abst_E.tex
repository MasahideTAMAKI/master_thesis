\begin{center}
\hrulefill\\
\vskip\baselineskip
{\bf\Large Characterization and Control of Cryogenic Suspension in KAGRA}\addcontentsline{toc}{chapter}{Abstract} \\
\large{\qquad\qquad\qquad\qquad\qquad\qquad\qquad\qquad\qquad\qquad\qquad\qquad\qquad\qquad\qquad\,\,\, by Masahide Tamaki}\\
\large{                             \qquad\quad\,\,\,\,Supervisor: Shinji Miyoki}\\
\vskip\baselineskip
{\bf \Large Abstract}
\end{center}
{\fontsize{10.2pt}{20pt}\selectfont
\qquad In September 2015, LIGO, a gravitational-wave observatories in America, succeeded in directly detecting gravitational waves for the first time. Since then, the international gravitational wave observation network has been established through joint observations by LIGO and Virgo, a gravitational-wave detector in Italy and about 90 gravitational wave events have been detected by January 2023. The gravitational wave astronomy thus opened is expected to developing as the international gravitational wave observation network expands further. KAGRA, a Japanese gravitational wave telescope, aims to join the international gravitational wave observation network, and contribute the improvement of the accuracy in determining the direction of the wave source and the observation of the polarization of gravitational waves. In April 2020, KAGRA conducted its first joint observation (O3GK) with GEO 600, a gravitational wave telescope in Germany, and will conduct a joint observation (O4) with LIGO and Virgo in March 2023.\\
\qquad KAGRA has a unique feature of cooling the sapphire mirror down to 20 K to reduce thermal noise, which is not found in past gravitational wave detectors, and the mirror will be cooled to achieve a target sensitivity of 10 Mpc in O4. However, unlike O3GK, where observations were made with the mirror at room temperature, the cooling of the mirror changes the properties of the suspension system, which may cause some anomalies or break down the control system that has worked well at room temperature. Moreover, the damping control is necessary to suppress the vibration for the stable operation of the interferometer, but the sensitivity of the detector in O3GK was limited by the noise from the control of cryogenic suspension system in the low frequency band of $10\sim50$ Hz. This frequency band corresponds to the frequency of the inspiral phase of gravitational waves from neutron star binary, and the neutron star mass can be obtained from the waveform. The neutron star masses thus obtained, together with the neutron star structure derived from the gravitational waves during the merger phase and later, are considered to be able to put a limit on the equation of state of neutron star matter. The realization of such a method will lead to the development of a wide range of fields such as astronomy, astrophysics, nuclear physics, and particle physics. In order to realize it, the reduction of control noise at low frequencies is required in O4.\\
\qquad Then, we investigated how the output of the photosensor, transfer function, and Q-value change for one of the cryogenic suspensions cooled down to 82 K as of December 2022. We have also designed a damping control for the cryogenic suspension to satisfy the requirement for damping in KAGRA, and have examined whether the damping control is valid at low temperatures or not. Furthermore, a new low-noise control filter (observation filter) was designed and implemented for use in the observation phase.\\
\qquad As a result, no abnormality was found in the cryogenic suspension system due to cooling. It is also confirmed that the damping control designed at room temperature is valid at low temperatures as long as the gain is tuned. Furthermore, by using a newly designed and implemented observation filter, the control noise of the cryogenic suspension at $10\sim50$ Hz was reduced by $2\sim3$ orders of magnitude, and the detector was successfully operated stably.}
\clearpage