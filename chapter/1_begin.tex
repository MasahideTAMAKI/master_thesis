\section{研究背景}
重力波とは, 物体の加速度運動によって生じた時空の歪みが波として光速で伝播する現象である. 1916年にA. Einsteinが一般相対性理論を用いてその存在を予言し\cite{1,2}, 1974年にR. A. HulseとJ. H. Taylorらが発見した連星パルサーPSR B1913+16の公転周期の観測によって存在が間接的に証明された\cite{3,4}. そして2015年9月14日, アメリカの重力波検出器LIGO\cite{LIGO}がブラックホール連星合体に伴う重力波をとらえ, 初の重力波直接検出を果たした\cite{5}. これは人類が宇宙を観測する新たな手段を手に入れたという点で非常に有意義なものであり, またこれまでよりも強い重力環境における一般相対性理論の検証も可能になった\cite{6,7}. さらにその2年後の2017年8月17日にはLIGOとイタリアの重力波検出器Virgo\cite{Virgo}が中性子星連星合体に伴う重力波を初検出し\cite{8}, 同月には中性子星連星合体からの重力波とショートガンマ線バー ストが同時観測され\cite{9}, そして電磁波によるフォローアップ観測で母天体が特定された\cite{10}. これは重力波を含めたマルチメッセンジャー天文学の始まりであり, 電磁波・ニュートリノ観測に重力波観測を組み合わせることによって, 天文学は新しい時代に突入した\cite{11}. \\
\quad LIGOやVirgo, さらに日本のKAGRA\cite{13}といった地上の大型重力波検出器はレーザー干渉計型重力波検出器と呼ばれる. これらはMichelson干渉計を基本とし, 基線長の潮汐的な微小変位を計測することで 重力波を検出する\cite{LIGO}. このうちKAGRAは2020年4月にドイツの重力波望遠鏡GEO 600\cite{GEO}との共同観測 (O3GK) を行ない, さらに2023年3月からはLIGOとVirgoとのO4 (Observation 4) 観測に参加することを予定している. \\
\quad さらにKAGRAは地下(岐阜県飛騨市神岡町)に建設され\cite{14}, またサファイア製の鏡を20 Kという極低温まで冷やす\cite{15}という点で先進的な重力波検出器であり, これらの技術はEinstein Telescopeなど次世代の重力波検出器にも応用される\cite{16}. ここで, KAGRAが地下へ建設されたのは地下の方が地上に比べて地面振動が1/100 程度であり, 地面振動の影響を抑えることができるからである. そして, さらに地面振動の影響を抑えるため, KAGRAでは多段振り子を用いて懸架装置で鏡を吊るしている. また, KAGRAが鏡を極低温に冷却するのは, 鏡や懸架装置を構成する原子の熱振動(ブラウン運動)を抑え, 熱雑音を低減することができるからである. そして, さらに熱雑音を低減するため, 低温において高い機械的Q値を示すサファイア製の鏡(Q$\sim 10^8$)を同じくサファイア製のファイバー(Q$\sim 10^7$)で吊るして冷却している\cite{18}. \\
\quad 低温懸架装置に関わるO3GKでの問題点は大きく2つある. 1つ目の問題として, O3GKでは鏡が常温の状態で観測が行われたため, 低温懸架装置の冷却によって起こりうる問題点の検証ができなかったことが挙げられる\cite{PTEP}. よって, 低温にすることで懸架装置の特性がどのように変化するかということを調べるのは重要な課題の1つである. もう1つの問題としては, $10\sim50$ Hzの周波数帯において, 低温懸架装置の制御雑音が感度を制限していたことが挙げられる\cite{PTEP}. KAGRAを重力波検出器として作動させるためには, 低温懸架装置の位置や姿勢を制御する必要があるが, その制御由来の雑音が$10\sim50$ Hz程度で支配的となっていた. よって, 低温懸架装置の制御雑音の低減は重要な課題である.

\section{本研究の主題・論文の構成}
次回のO4観測では感度向上のために懸架装置を極低温まで冷却して運転する. しかし, 冷却する際に物性が変わることによって, 懸架装置に異常が生じたり, 伝達関数やQ値などが変化して制御に変更の必要性が生じる可能性がある. そこで, 現在80 Kほどまで冷却された低温懸架装置において, 室温と低温における低温懸架装置の特性(フォトセンサの出力, 伝達関数, Q値)を比較した. \\
\quad また, 懸架装置の制御は重力波検出器の安定な運用に欠かせないものである. 重力波検出器では多段振り子を用いることで, 共振周波数より高い周波数において地面振動の影響を抑えている. しかし, 振り子の共振周波数では鏡に伝わる地面の振動が増幅されてしまう. そこで, 懸架装置の変位を局所的にセンサで検出し, 速度に比例した力をアクチュエータを通じてフィードバックすることによって振動を減衰させるダンピング制御を行う. この制御によって, 地震などの大きな外乱があっても, 共振モードを素早く抑えて干渉計をロック(共振器を共振状態に保つことをそう呼ぶ)することができ, 観測時間を伸ばすことに繋がる. 本研究ではサファイア鏡を吊るした低温懸架装置について, KAGRAにおける要求を満たすようなダンピング制御を行った. \\
\quad 一方, ダンピング制御を実装することによってO3GKで問題となったように, $10\sim50$ Hzの低周波数帯で制御雑音が感度を制限してしまう. これはダンピング制御のようなフィードバック制御では, 外乱を抑制しようとすると制御雑音が大きくなってしまうというジレンマが存在するからである. しかし, 干渉計が一度ロックした後は, それほど強い制御が必要なわけではない. よって, 干渉計が安定に動作する限り, なるべく雑音が生じない制御に移行することが望ましい. そこで, 観測状態で用いる制御として, 制御雑音を可能な限り生まないようなものを設計し実装した. その結果, 現在のKAGRAの感度に対して低温懸架装置の制御雑音が十分小さく, またその制御を用いて干渉計が安定に動作することを確認した. \\\\
\noindent
\quad 本研究で制御雑音を低減した$10\sim50$ Hzという低周波は, KAGRAが観測対象としている中性子星連星からの重力波のうち, インスパイラルフェイズの周波数にあたる. インスパイラルフェイズとは, 連星がお互いに共通重心の周りを公転しながら近づいていく段階であり, ポスト・ニュートン近似によるインスパイラルフェイズの波形を具に観測することで中性子星の質量が得られる\cite{compact}. こうして得られた中性子星の質量は, 合体フェイズ以降の重力波から得られる中性子星の構造と合わせることで, 中性子星物質の状態方程式に制限をつけることができると考えられている. それが実現すれば, 宇宙物理学や天文学だけでなく, 素粒子物理や原子核物理など幅広い分野の発展につながる.\\\\
\noindent
\quad 本論文ではまず, 第\ref{第2章}章で一般相対性理論を用いて重力波を導出し, その基本的な性質を述べる. また, レーザー干渉計型重力波望遠鏡の原理や雑音源についてまとめる. 第\ref{第3章}章では懸架装置による防振の原理とKAGRAにおける防振装置について述べる. 第\ref{第4章}章ではKAGRAのType-A Suspensionについて, 主に低温懸架装置の機構と用いられるセンサ・アクチュエータについて詳しく紹介し, その冷却法について述べる. 第\ref{第5章}章では低温懸架装置の特性評価について測定結果と考察をまとめる. 第\ref{第6章}章では低温懸架装置のダンピング制御についてその必要性と原理について詳しく述べる. そして設計した制御フィルタを紹介し, KAGRAにおける要求を満たす制御が行えたことを示す. 第\ref{第7章}章ではフィードバック制御において外乱の抑制と雑音の低減がトレードオフの関係にあることを示す. その後, 観測段階で用いる, 低雑音かつ安定な制御の設計とその効果についてまとめる. そして, 第\ref{第8章}章で本研究のまとめと今後の展望を記す. 
